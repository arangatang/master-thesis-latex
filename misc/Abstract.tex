%\thispagestyle{empty}   - 
%For tips on writing this section, refer \cite{wallwork_177}
Time series forecasting is an important tool for businesses, organizations and governments as it enables them to plan for the future. Historically, deep learning based forecasting models have been shown to perform poorly compared to classical approaches such as ARIMA, ETS and Theta. In recent years several deep learning based methods has been proposed and thus comparing these fairly and accurately is important. This thesis investigate several state of the art forecasting models, and methods for how to perform fair, accurate and reproducible comparisons. The domain of machine learning research is currently in a reproducibility crisis, thus particular focus is placed on identifying methods for generating technically, statistically and conceptually reproducible results.

This thesis also introduces Crayon, an open source benchmarking suite for fair, accurate and reproducible comparisons of forecasting methods. Crayon makes use of distributions of error metrics for its benchmarks and ranks algorithms against each other through a novel aggregation method of error distributions. This method, named RMSE4D, is applicable to any error metric distribution and scores algorithms generating consistent forecasts higher. Furthermore, tooling based on the Kolmogorov-Smirnov 2 sample test is employed in Crayon to enable statistical reproducibility of benchmarks and its practical benefits are demonstrated through protecting a forecasting framwork against accuracy regressions. Crayon is used in this thesis to benchmark 13 state of the art forecasting algorithms on four popular public dataset with complimentary characteristics in terms of trend \& seasonality. The major findings of this comparison is that DeepAR, Transformer and the N-BEATS Ensamble are the best performing models on three of the datasets but that naive and classical approaches such as Theta are the best performing algorithms for highly trended datasets.

\noindent
%This template is intended to give an introduction of how to write diploma and master thesis at the chair 'Architektur der Vermittlungsknoten' of the Technische Universit�t Berlin. Please don't use the term 'Technical University' in your thesis because this is a proper name. 
%\\
%\\
%On the one hand this PDF should give a guidance to people who will soon start to write their thesis. The overall structure is explained by examples. On the other hand this text is provided as a collection of LaTeX files that can be used as a template for a new thesis. Feel free to edit the design.
%\\
%\\
%It is highly recommended to write your thesis with LaTeX. I prefer to use Miktex in combination with TeXnicCenter (both freeware) but you can use any other LaTeX software as well. For managing the references I use the open-source tool jabref. For diagrams and graphs I tend to use MS Visio with PDF plugin. Images look much better when saved as vector images. For logos and 'external' images use JPG or PNG. In your thesis you should try to explain as much as possible with the help of images.
%\\
%\\
%The abstract is the most important part of your thesis. Take your time to write it as good as possible. Abstract should have no more than one page. It is normal to rewrite the abstract again and again, so  probaly you won't write the final abstract before the last week of due-date. Before submitting your thesis you should give at least the abstract, the introduction and the conclusion to a native english speaker. It is likely that almost no one will read your thesis as a whole but most people will read the abstract, the introduction and the conclusion.
%\\
%\\
%Start with some introductionary lines, followed by some words why your topic is relevant and why your solution is needed concluding with 'what I have done'. Don't use too many buzzwords. The abstract may also be read by people who are not familiar with your topic.